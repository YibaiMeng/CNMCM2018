% !Mode:: "TeX:UTF-8"
%!TEX program  = xelatex

\documentclass{cumcmthesis}
%\documentclass[withoutpreface,bwprint]{cumcmthesis} %去掉封面与编号页

\usepackage{url}
\title{(2018B)智能RGV的动态调度策略}
\tihao{B}
\baominghao{01001015}
\schoolname{北京大学}
\membera{包慧语}
\memberb{孟逸白}
\memberc{吴越}
\supervisor{}
\yearinput{2018}
\monthinput{09}
\dayinput{16}

\begin{document}

\maketitle
\begin{abstract}

本文是对智能RGV的动态调整策略问题的研究,通过在模拟器上对不同的调度算法验证其性能并分析,得出一般性的结论。

第一问中,我们首先通过尝试一个简单版本的调度策略,并在此基础上尝试不同的优化方案,比较其优劣,{\color{red}{发现xxxx}}

第二问中,{\color{red}{xxxx}}

在前两问中,由于机器的行为是可预测的,对于细节的调优极易陷入参数的深度依赖中,而第三问带有随机过程,可以较好地评估一个方案的优劣。我们通过预先生成一个机器失败的序列,通过比较一个动态算法(即对未来不可知的算法)和静态算法之间的差距,来评估我们算法的优劣,

\keywords{调度算法\quad  随机过程\quad   动态策略\quad  {\color{red}{还有啥}}}
\end{abstract}

%目录
\tableofcontents

\section{问题重述}



\subsection{问题的提出}



\section{模型的假设}

\begin{itemize}
\item 对于第一问和第二问,所有的时间可预测,决策可以使用未来的信息进行推断
\item 在第三问中机器的损坏是随机并且不可预测的,所以决策系统只知道此机器损坏的消息,同时不知道什么时候能够修好
\end{itemize}

\section{符号说明}
\begin{center}
\begin{tabular}{cc}
 \hline
 \makebox[0.3\textwidth][c]{符号}	&  \makebox[0.4\textwidth][c]{意义} \\ \hline
 D	    & 木条宽度(cm) \\ \hline
 L	    & 木板长度(cm)  \\ \hline
 W	    & 木板宽度(cm)  \\ \hline
 N	    & 第n根木条  \\ \hline
 T	    & 木条根数  \\ \hline
\end{tabular}
\end{center}

\section{问题分析}

\subsection{问题一分析}
题目要求建立模型描述折叠桌的动态变化图,由于在折叠时用力大小的不同,我们不能描述在某一时刻折叠桌的具体形态,但我们可以用每根木条的角度变化来描述折叠桌的动态变化。首先,我们知道折叠桌前后左右对称,我们可以运用几何知识求出四分之一木条的角度变化。最后,根据初始时刻和最终形态两种状态求出桌腿木条开槽的长度。




问题流程图:
\begin{figure}[!h]
\centering
\caption{问题三流程图}
\end{figure}

\section{绘制普通三线表格}
表格应具有三线表格式,因此常用 booktabs宏包,其标准格式如表~\ref{tab001}~所示。
\begin{table}[!htbp]
\caption{标准三线表格}\label{tab001} \centering
\begin{tabular}{ccccc}
\toprule[1.5pt]
$D$(in) & $P_u$(lbs) & $u_u$(in) & $\beta$ & $G_f$(psi.in)\\
\midrule[1pt]
 5 & 269.8 & 0.000674 & 1.79 & 0.04089\\
10 & 421.0 & 0.001035 & 3.59 & 0.04089\\
20 & 640.2 & 0.001565 & 7.18 & 0.04089\\
\bottomrule[1.5pt]
\end{tabular}
\end{table}

其绘制表格的代码及其说明如下。
\begin{tcode}
\begin{table}[!htbp]
\caption[标签名]{中文标题}
\begin{tabular}{cc...c}
\toprule[1.5pt]
表头第1个格   & 表头第2个格   & ... & 表头第n个格  \\
\midrule[1pt]
表中数据(1,1) & 表中数据(1,2) & ... & 表中数据(1,n)\\
表中数据(2,1) & 表中数据(2,2) & ... & 表中数据(2,n)\\
...................................................\\
表中数据(m,1) & 表中数据(m,2) & ... & 表中数据(m,n)\\
\bottomrule[1.5pt]
\end{tabular}
\end{table}
\end{tcode}

\bigskip
table环境是一个将表格嵌入文本的浮动环境。
tabular环境的必选参数由每列对应一个格式字符所组成:c表示居中,l表示左对齐,r表示右对齐,其总
个数应与表的列数相同。此外,\verb|@{文本}|可以出现在任意两个上述的列格式之间,其中的文本将被插入每一行
的同一位置。表格的各行以\verb|\\|分隔,同一行的各列则以\&分隔。
\verb|\toprule|、\verb|\midrule|和\verb|\bottomrule|三个命令是由booktabs宏包提供的,其
中\verb|\toprule|和\verb|\bottomrule|分别用来绘制表格的第一条(表格最顶部)和第三条(表格最底部)水平线,
\verb|\midrule|用来绘制第二条(表头之下)水平线,且第一条和第三条水平线的线宽为1.5pt,第二条水平线的线宽为1pt。
引用方法:“如表~\verb|\ref{标签名}|~所示”。


%参考文献
\begin{thebibliography}{9}%宽度9
 \bibitem{bib:one} ....
 \bibitem{bib:two} ....
\end{thebibliography}

\newpage
%附录
\begin{appendices}
\section{RGV行为模拟器--Python源代码}
%\begin{lstlisting}[language=python]
%\input{../simulator.py}
%\end{lstlisting}
\lstinputlisting[language=python]{../simulator.py}
\section{规划解决程序--lingo源代码}
\begin{lstlisting}[language=c]
kk=2;
[mdd,ndd]=size(dd);
while ~isempty(V)
    [tmpd,j]=min(W(i,V));tmpj=V(j);
for k=2:ndd
    [tmp1,jj]=min(dd(1,k)+W(dd(2,k),V));
    tmp2=V(jj);tt(k-1,:)=[tmp1,tmp2,jj];
end
    tmp=[tmpd,tmpj,j;tt];[tmp3,tmp4]=min(tmp(:,1));
if tmp3==tmpd, ss(1:2,kk)=[i;tmp(tmp4,2)];
else,tmp5=find(ss(:,tmp4)~=0);tmp6=length(tmp5);
if dd(2,tmp4)==ss(tmp6,tmp4)
    ss(1:tmp6+1,kk)=[ss(tmp5,tmp4);tmp(tmp4,2)];
else, ss(1:3,kk)=[i;dd(2,tmp4);tmp(tmp4,2)];
end;
end
    dd=[dd,[tmp3;tmp(tmp4,2)]];V(tmp(tmp4,3))=[];
    [mdd,ndd]=size(dd);
    kk=kk+1;
end;
S=ss;
D=dd(1,:);
 \end{lstlisting}
\end{appendices}

\end{document} 